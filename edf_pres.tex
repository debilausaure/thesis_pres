\documentclass[xetex,notheorems,hyperref={pdfpagelabels=true},xcolor={table},aspectratio=169]{beamer}
\usetheme{minflat}
\usepackage{booktabs}
\usepackage[scale=2]{ccicons}

\usepackage[style=american]{csquotes}

\usetikzlibrary{decorations.pathreplacing, decorations.pathmorphing,calc,arrows,positioning,patterns,fadings,shadows, arrows.meta}
\tikzset{
    %Define standard arrow tip
    >=stealth'
}


\tikzfading[name=fade right, left color=transparent!0, right color=transparent!100]
\tikzfading[name=fade left, left color=transparent!100, right color=transparent!0]
\tikzfading[name=fade out, inner color=transparent!0, outer color=transparent!100]

%define hatched pattern
\pgfdeclarepatternformonly{south west lines}{\pgfqpoint{-0pt}{-0pt}}{\pgfqpoint{3pt}{3pt}}{\pgfqpoint{3pt}{3pt}}{
        \pgfsetlinewidth{0.4pt}
        \pgfpathmoveto{\pgfqpoint{0pt}{0pt}}
        \pgfpathlineto{\pgfqpoint{3pt}{3pt}}
        \pgfpathmoveto{\pgfqpoint{2.8pt}{-.2pt}}
        \pgfpathlineto{\pgfqpoint{3.2pt}{.2pt}}
        \pgfpathmoveto{\pgfqpoint{-.2pt}{2.8pt}}
        \pgfpathlineto{\pgfqpoint{.2pt}{3.2pt}}
        \pgfusepath{stroke}}

\setbeamercovered{invisible}

%%% enable notes on second screen
%\usepackage{pgfpages}
%\setbeameroption{show notes on second screen=right}
%\setbeamertemplate{note page}[compress]

\definecolor{proofgreen}{RGB}{233,4,105}
\definecolor{implementation}{RGB}{41,2,81}
	
%%%%%%%%%%%%%%%%%%%%%%%%%%%%%%%%%%%%%%%%%%%%%%%%%%%
%%%%	define content of title page
%%%%%%%%%%%%%%%%%%%%%%%%%%%%%%%%%%%%%%%%%%%%%%%%%%%
\def\talkTitle{Formal Correctness Proof for an EDF Scheduler Implementation}
\def\talkSubtitle{RTAS 2022}
\def\talkKeywords{Keywords}

%% Define meta data of pdf
\hypersetup{
    pdftitle={Slides to talk - \talkTitle},
	pdfsubject={\talkTitle},
	pdfauthor={Florian Vanhems},
	pdfkeywords={\talkKeywords},
	colorlinks=false
}


%%%%%%%%%%%%%%%%%%%%%%%%%%%%%%%%%%%%%%%%%%%%%%%%%%%
%%%%	set content of title page
%%%%%%%%%%%%%%%%%%%%%%%%%%%%%%%%%%%%%%%%%%%%%%%%%%%
\title{\talkTitle}  
\subtitle{\textbf{\talkSubtitle}} 
\author{Florian Vanhems -- florian.vanhems@univ-lille.fr}
\DTMlangsetup[en-GB]{ord=raise,monthyearsep={,\space}}
\date{\DTMtoday}
\institute{%
	\centering
	\def\svgwidth{.25\textwidth}
	\input{./images/logo_ulille.pdf_tex}
}

%%%%%%%%%%%%%%%%%%%%%%%%%%%%%%%%%%%%%%%%%%%%%%%%%%%
%%%%	theorem tools, theorem and proof styles
%%%%%%%%%%%%%%%%%%%%%%%%%%%%%%%%%%%%%%%%%%%%%%%%%%%
\setbeamertemplate{theorem}[ams style]
%\setbeamertemplate{theorems}[numbered]

\newcounter{chapter}
\setcounter{chapter}{1}
\theoremstyle{plain}
\newtheorem{theorem}{Theorem}[section]
\newtheorem{lemma}[theorem]{Lemma}
\newtheorem{proposition}[theorem]{Proposition}
\newtheorem{corollary}[theorem]{Corollary}

\theoremstyle{definition}
\newtheorem{conclusion}[theorem]{Conclusion}
\newtheorem*{definition}{Definition}
\newtheorem*{remark}{Remark}
\newtheorem*{dummyblock}{dummyblock}

\theoremstyle{example}
\newtheorem{example}[theorem]{Example}

\newenvironment<>{dummyblock}[1]{%
	\setbeamercolor{block title}{fg=white,bg=white}%
	\setbeamercolor{block body}{fg=normal text.fg,bg=white}%
	\begin{block}#2{#1}}{\end{block}}

\begin{document}
\renewcommand{\leq}{\leqslant}
\renewcommand{\geq}{\geqslant}

\renewcommand\theta\vartheta


%%%%%%%%%%%%%%%%%%%%%%%%%%%%%%%%%%%%%%%%%%%%%%%%%%%
%%%%	title page
%%%%%%%%%%%%%%%%%%%%%%%%%%%%%%%%%%%%%%%%%%%%%%%%%%%
\begin{frame}[plain]
	\titlepage
\end{frame}


%%%%%%%%%%%%%%%%%%%%%%%%%%%%%%%%%%%%%%%%%%%%%%%%%%%
%%%%	toc
%%%%%%%%%%%%%%%%%%%%%%%%%%%%%%%%%%%%%%%%%%%%%%%%%%%
\begin{frame}
	\tableofcontents
\end{frame}

\begin{frame}[fragile]
	\frametitle{Motivations}
	\begin{itemize}
		\item{Critical software of embedded systems}
		\item{EDF is optimal and well understood}
		\item{Experiment with our proof methodology}
	\end{itemize}
\end{frame}

\begin{frame}[fragile]
	\frametitle{Aspirations}
	\begin{itemize}
		\item{Proof of concept - not a production ready scheduler}
		\item{Share our conception and understanding of software proofs}
	\end{itemize}
\end{frame}


%%%%%%%%%%%%%%%%%%%%%%%%%%%%%%%%%%%%%%%%%%%%%%%%%%%
%%%%	Overview
%%%%%%%%%%%%%%%%%%%%%%%%%%%%%%%%%%%%%%%%%%%%%%%%%%%

\section{Overview}

\highlightedFrame[First formally proven implementation\\of an Earliest Deadline First scheduler for arbitrary sequences of jobs]{}%Claim\\[1cm]}

\begin{frame}[fragile]
	\frametitle{Overview of the scheduler}

	\begin{center}
\begin{tikzpicture}[>=triangle 45,font=\sffamily, every text node part/.style={align=center}, every node/.style={transform shape}]

	\node (election) at (0, 1.5) [fill=proofgreen, draw, semithick, minimum height=2cm, minimum width=4cm] {};
	\node at (election) [fill=white, draw, semithick, minimum height=1.75cm, minimum width=3.75cm] {Election Function};

	\node (state) at (1.5, -1.5) [minimum height=2cm, minimum width=2cm] {};
	\draw[semithick,fill=implementation] (state.90)--(state.45)--(state.0)--(state.180)--(state.225)--(state.270)--cycle;
	\draw[semithick,fill=proofgreen] (state.90)--(state.135)--(state.180)--(state.center)--(state.270)--(state.315)--(state.0)--(state.center)--cycle;
	\node at (state) [fill=white, draw, semithick, minimum height=2cm, minimum width=1.8cm] {State};

	\node (interface) at (-1.5, -1.5) [minimum height=2cm, minimum width=2cm] {};
	\draw[semithick,fill=implementation] (interface.90)--(interface.45)--(interface.0)--(interface.180)--(interface.225)--(interface.270)--cycle;
	\draw[semithick,fill=proofgreen] (interface.90)--(interface.135)--(interface.180)--(interface.center)--(interface.270)--(interface.315)--(interface.0)--(interface.center)--cycle;
	\node at (interface) [fill=white, draw, semithick, minimum height=2cm, minimum width=1.8cm] {Interface};

	\node[draw, semithick, minimum height = 5cm, minimum width = 3cm, fill=implementation] (backend) at (-5, 0) {Back-end};
	\node at (backend) [fill=white, draw, semithick, minimum height = 5cm, minimum width=2.75cm] {Back-end};

\end{tikzpicture}
\end{center}

\end{frame}

\begin{frame}[fragile]
	\frametitle{Star of the talk}

	\begin{center}
\begin{tikzpicture}[>=triangle 45,font=\sffamily, every text node part/.style={align=center}, every node/.style={transform shape}]

	\node (election) at (0, 1.5) [fill=proofgreen, draw, semithick, minimum height=2cm, minimum width=4cm] {};
	\node at (election) [fill=white, draw, semithick, minimum height=1.75cm, minimum width=3.75cm] {Election Function};

	\onslide<2- > {
		\node at (-5, -2) {Written directly in Coq};
	}
	\onslide<3- > {
		\node at (0, -2) {Formally proven properties};
	}
	\onslide<4- > {
		\node at (5, -2) {Translated word to word to C};
	}

\end{tikzpicture}
\end{center}
	
\end{frame}

\begin{frame}[fragile]
	\frametitle{General informations about the scheduler}

	\begin{itemize}
		\item{Schedules arbitrary sequences of jobs (as opposed to \emph{tasks})}
		\item{Periodically called}
		\item{Online \footnote{: while the election function schedules online, our scheduler feeds it hard coded jobs for simplicity}}
	\end{itemize}
\end{frame}

\begin{frame}[fragile]
	\frametitle{Election function of an Earliest Deadline First scheduler}

	\begin{center}

\definecolor{job1color}{RGB}{163,137,198}
\definecolor{job2color}{RGB}{233,4,105}

\begin{tikzpicture}[>=triangle 45,font=\sffamily, every text node part/.style={align=center}, every node/.style={transform shape}]

	%timeline
	\node (tstart) at (-2, 0) {};
	\node (tend) at (12, 0) {};
	\draw[->] (tstart.center) -- (tend.center);
	\node[below=0.2cm of tend] {time (a.u.)};

	% floating job
	\onslide<1-5> {
		\node[draw, fill=job1color, minimum height=1cm, minimum width=4cm] (job) at (4.5, 3.5) {};
	}

	\onslide<1>{
		\node[above=0.2cm of job] {Job \small{$a$}};
	}

	% budget marker
	\onslide<2-5> {
		\draw[<->] (2.5, 4.2) -- (6.5, 4.2);
		\node[above=0.2cm of job] {budget ($c_a$)};
	}

	% release marker
	\onslide<3-7> {
		\node (tjobarrival) at (2, 2.5) {$t = 2$};
		\node[above left=0.3cm of tjobarrival] (releasetext) {release date ($r_a$)};
		\draw[->, -{Latex[length=1mm,width=1mm]}] (releasetext) to [out=270,in=180] (tjobarrival);
		\draw[dashed] (tjobarrival) -- ($(tstart)!(tjobarrival)!(tend)$);
	}

	% deadline marker
	\onslide<4-7> {
		\node (tjobdeadline) at (7, 2.5) {$t = 7$};
		\node[above right=0.3cm of tjobdeadline] (deadlinetext) {deadline ($d_a$)};
		\draw[->, -{Latex[length=1mm,width=1mm]}] (deadlinetext) to [out=270,in=0] (tjobdeadline);
		\draw[dashed] (tjobdeadline) -- ($(tstart)!(tjobdeadline)!(tend)$);
	}

	% available execution time
	\onslide<5-> {
		\node (job1height) at (2, 1) {};

		\draw (tjobarrival |- job1height) -- (tjobdeadline |- tend) node [midway, draw, fill=white, minimum width = 5cm, minimum height = 1cm] (execperiod) {};
		\node[draw, fill=white, pattern=south west lines, minimum width = 5cm, minimum height = 1cm] at (execperiod) {};
	}

	\onslide<5> {
		\node[above= 0.2cm of execperiod] {Job's execution period};
	}

	\onslide<6> {
		\node (tjobduration1) at (6, 2) {$t = 6$};
		\node[draw, fill=job1color, minimum width = 4cm, minimum height = 1cm] (job1) at (4, 0.5) {};
		\node[below=0.2cm of $(tstart)!(tjobarrival)!(tend)$] (startduration1) {}; 
		\node[below=0.2cm of $(tstart)!(tjobduration1)!(tend)$] (endduration1) {}; 
		\draw[<->] (startduration1.center) -- (endduration1.center) node [midway,below] {budget ($c_a$)};
		\draw[dashed] (tjobduration1) -- ($(tstart)!(tjobduration1)!(tend)$);
	}

	\onslide<7> {
		\node (tjobduration2) at (3, 2) {$t = 3$};
		\node[draw, fill=job1color, minimum width = 4cm, minimum height = 1cm] (job2) at (5, 0.5) {};
		\node[below=0.2cm of $(tstart)!(tjobduration2)!(tend)$] (startduration2) {}; 
		\node[below=0.2cm of $(tstart)!(tjobdeadline)!(tend)$] (endduration2) {}; 
		\draw[<->] (startduration2.center) -- (endduration2.center) node [midway,below] {budget ($c_a$)};
		\draw[dashed] (tjobduration2) -- ($(tstart)!(tjobduration2)!(tend)$);
	}

	%% Second part with the new job

	\onslide<8-11> {
		\node (tjobarrival2) at (2, -0.5) {$r_a$};
		\draw[dashed] (tjobarrival2) -- ($(tstart)!(tjobarrival2)!(tend)$);

		\node (tjobdeadline2) at (7, -0.5) {$d_a$};
		\draw[dashed] (tjobdeadline2) -- ($(tstart)!(tjobdeadline2)!(tend)$);
	}

	\onslide<8-10> {
		\node[draw, fill=job1color, minimum width = 4cm, minimum height = 1cm] (job3) at (4, 0.5) {};
	}

	\only<8-9> {
		\node[draw, fill=job2color, minimum width = 1cm, minimum height = 1cm] (job4) at (4.5, 3.5) {};
		\node[above=0.2cm of job4] {Job \small{$b$}};
	}

	\onslide<9-11> {
		% Red release
		\node (tjobarrival2) at (3, 2.5) {};
		\draw[dashed] (tjobarrival2) -- ($(tstart)!(tjobarrival2)!(tend)$);

		% Red deadline
		\node (tjobdeadline2) at (6, 2.5) {};
		\draw[dashed] (tjobdeadline2) -- ($(tstart)!(tjobdeadline2)!(tend)$);

		% Red exec period
		\node (job2height) at (2, 2) {};
		\draw (tjobarrival2 |- job2height) -- (tjobdeadline2 |- job1height) node [midway, draw, fill=white, minimum width = 3cm, minimum height = 1cm] (execperiod2) {};
		\node[draw, pattern color=job2color, pattern=south west lines, minimum width = 3cm, minimum height = 1cm] at (execperiod2) {};
	}

	\onslide<9> {
		\node[above=0.0cm of tjobarrival2] (arrivalmarker) {$t = 3$};
		\node[above=0.0cm of tjobdeadline2] (deadlinemarker) {$t = 6$};
		\node[above=0.0cm of arrivalmarker] {$r_b$};
		\node[above=0.0cm of deadlinemarker] {$d_b$};
	}

	\onslide <10-11> {
		\node[above=0.0cm of tjobarrival2] {$r_b$};
		\node[above=0.0cm of tjobdeadline2] {$d_b$};
		\node (tjob2done) at (4, 2.5) {};
		\draw[dashed] (tjob2done) -- ($(tstart)!(tjob2done)!(tend)$);
	}

	\onslide<10-11> {
		\node[draw, fill=job2color, minimum width = 1cm, minimum height = 1cm] (job5) at (3.5, 1.5) {};
	}

	\onslide<10> {
		\node (deadlinecomp) at (4.5, 4) {$d_b < d_a$};
		\draw[->, -{Latex[length=1mm,width=1mm]}] (deadlinecomp) to [out=270,in=90] (job5);
		\node[below=0.15cm of job5, color=job2color] {\Huge{!}};
	}

	\onslide<11-> {
		\node[draw, fill=job1color, minimum width = 1cm, minimum height = 1cm] (job6) at (2.5, 0.5) {};
		\node[draw, fill=job1color, minimum width = 3cm, minimum height = 1cm] (job7) at (5.5, 0.5) {};
	}

	\onslide<12-> {
		\node[draw, fill=job2color, minimum width = 1cm, minimum height = 1cm] (job8) at (3.5, 0.5) {};
	}

	\onslide<12> {
		\node (t1) at (1,2) {$t=1$};
		\draw[dashed] (t1) -- ($(tstart)!(t1)!(tend)$);
		\node[above=0.0cm of t1] {$\text{election function}(1) = \varnothing$};
	}

	\onslide<13> {
		\node (t2) at (2,2) {$t=2$};
		\draw[dashed] (t2) -- ($(tstart)!(t2)!(tend)$);
		\node[above=0.0cm of t2] {$\text{election function}(2) = \text{Job}~a$};
	}

	\onslide<14> {
		\node (t3) at (3,2) {$t=3$};
		\draw[dashed] (t3) -- ($(tstart)!(t3)!(tend)$);
		\node[above=0.0cm of t3] {$\text{election function}(3) = \text{Job}~b$};
	}

	\onslide<15> {
		\node (t4) at (4,2) {$t=4$};
		\draw[dashed] (t4) -- ($(tstart)!(t4)!(tend)$);
		\node[above=0.0cm of t4] {$\text{election function}(4) = \text{Job}~a$};
	}
\end{tikzpicture}
\end{center}

\end{frame}

%%%%%%%%%%%%%%%%%%%%%%%%%%%%%%%%%%%%%%%%%%%%%%%%%%%
%%%%	Section proof
%%%%%%%%%%%%%%%%%%%%%%%%%%%%%%%%%%%%%%%%%%%%%%%%%%%
\section{Proofs \& Hypotheses}

\begin{frame}[fragile]
	\frametitle{So what did we prove?}
	\begin{center}
		\huge{Schedulable job sets are scheduled such that\\no job misses its deadline}
	\end{center}
\end{frame}

\begin{frame}[fragile]
	\frametitle{Schedulability property}

	Given any two moments $t, t'$, let $\Gamma_{t, t'}$ be the set of jobs $j$ to schedule in the interval $[t, t']$.\\
	If the sum of the budget $c_j$ of the jobs in that set is less than $t' - t$, then the job set is \emph{schedulable}.

	\begin{definition}[Schedulability property]
	\[
		\forall ~t, t'. ~t < t' \implies \sum_{j \in \Gamma_{t, t'}}c_j \leq t' - t
	\]
	\end{definition}
\end{frame}

\begin{frame}[fragile]
	\frametitle{Well-formedness assumptions}

	For each job :
	\begin{itemize}
		\item{$r_i + c_i \leq d_i$ : the deadline comes late enough for the job to complete its execution if executed alone on the processor}
		\item{$0 < \delta_j \leq c_j$ : the actual duration $\delta$ of a job is strictly positive and less than its budget $c$}
		\item{unique identifiers}
		\item{released exactly once}
	\end{itemize}
\end{frame}

\highlightedFrame[]{Three~~steps~~to~~reach correctness}

\begin{frame}[fragile]
	\frametitle{Earliest Deadline First policy}
	\begin{block}{EDF scheduling policy}
		For any job $j$ and any time instant $t$, if the job $j$ is running at instant $t$, then for any other job $j'$ that is ready to run at the same instant, it holds that $d_j \leq d_{j'}$.
	\end{block}

	\vspace{0.5cm}

	Applying the policy on a job set (up to a certain time instant $t$) is defined as :
	\begin{center}
		EdfPolicyUpTo $t$
	\end{center}
	
	\begin{block}{EDF policy correctnesss property}
	\vspace{-0.2cm}
	\[
		\text{schedulable} \implies \forall j. \forall t.~~\text{EdfPolicyUpTo}~t \implies \neg \text{overdue}~j~t
	\]
	\end{block}

\end{frame}

\begin{frame}[fragile]
	\frametitle{Correctness of an intermediate election function}

	Implement an idealised election function that acts like the one that will be executed.\\[0.5cm]

	The next step is to prove it implements the EDF policy defined previously.

	\begin{block}{Functional election function implements EDF policy}
	\vspace{-0.2cm}
	\[
		\forall t, \forall o, \forall s.~~
		\texttt{idealised\_scheduler}(t) = (o, s) \implies
	\text{EdfPolicyUpTo}~t.
	\]
	\end{block}

	\vspace{0.5cm}

	From this property follows the correctness of this idealised election function.

\end{frame}

\begin{frame}[fragile]
	\frametitle{Correctness of the election function}

	Implement the final, translatable to C, election function \emph{that relies on a chosen set of primitives}.\\[0.5cm]

	The next step is to prove that it acts the same way as the functional one.

	\begin{block}{Actual election function has same effects as functional}
		\vspace{-0.6cm}
		\begin{gather*}
			\forall t.\\
			\{
			    ~\textit{env} = E \land s = \textit{init}~
			\}\\
			(o, s') := \texttt{scheduler}~(t)\\
			\{
			    ~\texttt{idealised\_scheduler}~(t) = (o,s')~
			\}
		\end{gather*}
	\end{block}

	\vspace{0.5cm}

	From this property follows the correctness of the scheduler.

\end{frame}
%%%%%%%%%%%%%%%%%%%%%%%%%%%%%%%%%%%%%%%%%%%%%%%%%%%
%%%%
%%%%%%%%%%%%%%%%%%%%%%%%%%%%%%%%%%%%%%%%%%%%%%%%%%%
\section{Models \& Implementation}

\begin{frame}[fragile]
	\frametitle{Overview of the scheduler}

	\begin{center}
\begin{tikzpicture}[>=triangle 45,font=\sffamily, every text node part/.style={align=center}, every node/.style={transform shape}]

	\node (election) at (0, 1.5) [fill=proofgreen, draw, semithick, minimum height=2cm, minimum width=4cm] {};
	\node at (election) [fill=white, draw, semithick, minimum height=1.75cm, minimum width=3.75cm] {Election Function};

	\node (state) at (1.5, -1.5) [minimum height=2cm, minimum width=2cm] {};
	\draw[semithick,fill=implementation] (state.90)--(state.45)--(state.0)--(state.180)--(state.225)--(state.270)--cycle;
	\draw[semithick,fill=proofgreen] (state.90)--(state.135)--(state.180)--(state.center)--(state.270)--(state.315)--(state.0)--(state.center)--cycle;
	\node at (state) [fill=white, draw, semithick, minimum height=2cm, minimum width=1.8cm] {State};

	\node (interface) at (-1.5, -1.5) [minimum height=2cm, minimum width=2cm] {};
	\draw[semithick,fill=implementation] (interface.90)--(interface.45)--(interface.0)--(interface.180)--(interface.225)--(interface.270)--cycle;
	\draw[semithick,fill=proofgreen] (interface.90)--(interface.135)--(interface.180)--(interface.center)--(interface.270)--(interface.315)--(interface.0)--(interface.center)--cycle;
	\node at (interface) [fill=white, draw, semithick, minimum height=2cm, minimum width=1.8cm] {Interface};

	\node[draw, semithick, minimum height = 5cm, minimum width = 3cm, fill=implementation] (backend) at (-5, 0) {Back-end};
	\node at (backend) [fill=white, draw, semithick, minimum height = 5cm, minimum width=2.75cm] {Back-end};

\end{tikzpicture}
\end{center}

\end{frame}

\begin{frame}[fragile]
	\frametitle{The scheduler from the proof's point of view}
	\begin{center}
\begin{tikzpicture}[>=triangle 45,font=\sffamily, every text node part/.style={align=center}, every node/.style={transform shape}]

%	\node (scheduler) at (-2,0) [fill=white, draw, semithick, minimum height=6cm, minimum width=10cm]{};%, pattern=south west lines] {};
%	\node[above=0.2cm of scheduler] {\large{EDF Scheduler}};
	\onslide<1-> {
		\node (election) at (0, 1.5) [fill=proofgreen, draw, semithick, minimum height=2cm, minimum width=4cm] {};
		\node at (election) [fill=white, draw, semithick, minimum height=1.75cm, minimum width=3.75cm] {Election Function};

		\node (state) at (1.5, -1.5) [draw, semithick, minimum height=2cm, minimum width=2cm, fill=proofgreen] {};
%		\draw[semithick,fill=proofgreen] (state.90)--(state.135)--(state.180)--(state.center)--(state.270)--(state.315)--(state.0)--(state.center)--cycle;
		\node at (state) [fill=white, draw, semithick, minimum height=2cm, minimum width=1.8cm] {State};

		\node (interface) at (-1.5, -1.5) [draw, semithick, fill=proofgreen, minimum height=2cm, minimum width=2cm] {};
%		\draw[semithick,fill=proofgreen] (interface.90)--(interface.135)--(interface.180)--(interface.center)--(interface.270)--(interface.315)--(interface.0)--(interface.center)--cycle;
		\node at (interface) [fill=white, draw, semithick, minimum height=2cm, minimum width=1.8cm] {Interface};
		
	}

	\onslide<2> {
		\node (text) at (-6, 0) {\textbf{State monad}\\\emph{idealised} mathematical\\version of the components};
		\draw[->, -{Latex[length=1mm,width=1mm]}] (text) to [out=270, in=180] (interface);
		\draw[->, -{Latex[length=1mm,width=1mm]}] (text) to [out=0, in=146] (state);
	}

	\onslide<3-> {
		\node[minimum width = 3cm, minimum height = 3cm, shape=circle] (oracle) at (-5, 0){};
		\fill [color=proofgreen,path fading=fade out] (-7,-2) rectangle (-3,2);
		\node at (oracle) [fill=white, shape=circle, draw, semithick, minimum width = 1.8cm, minimum height = 1.8cm] {Oracles};
	}

	\onslide<4> {
		\node (text2) at (-7, -3) {\textbf{Environment monad}\\\emph{constraints} on the behaviour, no direct description};
		\draw[->, -{Latex[length=1mm,width=1mm]}] (text2) to [out=90, in=270] (oracle);
	}

\end{tikzpicture}
\end{center}

\end{frame}

\begin{frame}[fragile]
	\frametitle{The actual scheduler}
	\begin{center}
\begin{tikzpicture}[>=triangle 45,font=\sffamily, every text node part/.style={align=center}, every node/.style={transform shape}]

	\onslide<1-> {
		\node (election) at (0, 1.5) [fill=proofgreen, draw, semithick, minimum height=2cm, minimum width=4cm] {};
		\node at (election) [fill=white, draw, semithick, minimum height=1.75cm, minimum width=3.75cm] {Election Function};

		\node (state) at (1.5, -1.5) [draw, semithick, fill=implementation, minimum height=2cm, minimum width=2cm] {};
		\node at (state) [fill=white, draw, semithick, minimum height=2cm, minimum width=1.8cm] {State};

		\node (interface) at (-1.5, -1.5) [draw, semithick, fill=implementation, minimum height=2cm, minimum width=2cm] {};
		\node at (interface) [fill=white, draw, semithick, minimum height=2cm, minimum width=1.8cm] {Interface};

		\node[draw, semithick, minimum height = 5cm, minimum width = 3cm, fill=implementation] (backend) at (-5, 0) {Back-end};
		\node at (backend) [fill=white, draw, semithick, minimum height = 5cm, minimum width=2.75cm] {Back-end};
	}

	\onslide<2-> {
		\node (text) at (-2, -3.5) {\textbf{Actual} implementation\\Different from the model, but arguably close enough};
		\draw[->, -{Latex[length=1mm,width=1mm]}] (text) to [out=20, in=270] (state);
		\draw[->, -{Latex[length=1mm,width=1mm]}] (text) to [out=90, in=270] (interface);
		\draw[->, -{Latex[length=1mm,width=1mm]}] (text) to [out=150, in=270] (backend);
	}

\end{tikzpicture}
\end{center}

\end{frame}

\begin{frame}[fragile]
	\frametitle{Most common yet forgotten assumption}
	\begin{center}
		\huge{The models properly describe the behaviour\\ of components we rely on.}
	\end{center}
\end{frame}

%%%%%%%%%%%%%%%%%%%%%%%%%%%%%%%%%%%%%%%%%%%%%%%%%%%
%%%%	thanks slide, not in toc
%%%%%%%%%%%%%%%%%%%%%%%%%%%%%%%%%%%%%%%%%%%%%%%%%%%

\sectionNotInTocAndNavigation{Conclusion}

\begin{frame}{Conclusion}

	We have shown an EDF scheduler with a proved election function, describing :
	\begin{itemize}
		\item{The role of the election function, the interface and state, and the back-end}
		\item{The correctness of the election function}
		\item{The assumptions}
		\item{The monadic approach}
	\end{itemize}

\end{frame}
%%%%%%%%%%%%%%%%%%%%%%%%%%%%%%%%%%%%%%%%%%%%%%%%%%%
%%%%	conclusion frame
%%%%%%%%%%%%%%%%%%%%%%%%%%%%%%%%%%%%%%%%%%%%%%%%%%%
\begin{frame}{Thank you for your attention!}

	Sources \& directions to run the scheduler can be found on our repository :

	\begin{center}\url{https://github.com/2xs/pip_edf_scheduler}\end{center}

	and it passed the artifact validation process
%	\begin{center}
%		\tiny{These slides were created with the \href{https://github.com/vipowueb/minflat-beamer}{\emph{minflat beamer theme}} by Robert Baumgarth}.
%	\end{center}
\end{frame}

\end{document}
